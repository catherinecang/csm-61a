\documentclass{exam}
\usepackage{../../commonheader}
\lstset{language=Scheme}

%%% CHANGE THESE %%%%%%%%%%%%%%%%%%%%%%%%%%%%%%%%%%%%%%%%%%%%%%%%%%%%%%%%%%%%%%
\discnumber{8}
\title{\textsc{Streams, Iterators, and Generators}}
\date{April 10 to April 14, 2017}
%%%%%%%%%%%%%%%%%%%%%%%%%%%%%%%%%%%%%%%%%%%%%%%%%%%%%%%%%%%%%%%%%%%%%%%%%%%%%%%

\begin{document}
\maketitle
\rule{\textwidth}{0.15em}
\fontsize{12}{15}\selectfont

%%% INCLUDE TOPICS HERE %%%%%%%%%%%%%%%%%%%%%%%%%%%%%%%%%%%%%%%%%%%%%%%%%%%%%%%

\section{Iterators}
\begin{questions}
\begin{blocksection}
\question What is difference between an iterable and an iterator?
\begin{solution}[0.5in]
Iterator: Mutable object that tracks a position in a sequence, advancing on each call to \texttt{next} 

Iterable: Represents a sequence and returns a new iterator on each call to \texttt{iter}

To use in an English sentence:
Lists are "iterable". To go through a list, you make an object called an "iterator" to scan through the list.
\end{solution}
\end{blocksection}

\begin{blocksection}
\question \textbf{Accumulator} Write an iterator class that takes in a list and calculates the sum of the list thus far.
\begin{lstlisting}
>>> accu = Accumulator([1, 2, 3, 4, 5, 6])
>>> for a in accu:
            print(a)
1
3
6
10
15
21
\end{lstlisting}

\begin{solution}[1.5in]
\begin{lstlisting}
class Accumulator:
	def __init__(self, lst):
		self.lst = lst
		self.index = 0
		self.sum = 0
	def __next__(self):
		if self.index >= len(self.lst):
			raise StopIteration()
		self.sum += self.lst[self.index]
		self.index += 1
		return self.sum
	def __iter__(self):
		return self
\end{lstlisting}
\end{solution}
\end{blocksection}

\begin{blocksection}
\question Is this an iterator or an iterable or both?
\begin{solution}[0.5in]
Both; the \texttt{iter} method returns self and the \texttt{next} method is implemented. Note that an iterator is always an iterable, but an iterable is not always an iterator.
\end{solution}

\question (Optional) Write \texttt{Accumulator} so it works if it takes in any iterable, not just a list

\begin{solution}[1.5in]
\begin{lstlisting}
class Accumulator:
	def __init__(self, iterable):
    		self.iterable = iterable
		self.iterator = iter(iterable)
            self.sum = 0
	def __next__(self):
    	self.sum += next(self.iterator)
    		return self.sum
	def __iter__(self):
    		return self
\end{lstlisting}
\end{solution}
\end{blocksection}


\section{Generators}

%%% Question %%%
\begin{blocksection}
\question What does the following code block output?
\begin{lstlisting}
def foo():
	a = 0
	if a < 10:
		print("Hello")
		yield a
		print("World")

for i in foo():
	print(i)
\end{lstlisting}

\begin{solution}[0.75in]
\begin{lstlisting}
Hello
0
World
\end{lstlisting}
First time you call foo, it will yield a (which starts as 0)
\end{solution}

\question How can we modify \texttt{foo} so that \texttt{list(foo()) == [1, 2, 3, . . . , 10]}?  (You do not need to care about what gets printed)

\begin{solution}[0.75in]
Change the if to a while statement, and make sure to increment a. This looks like: 

\begin{lstlisting}
def foo():
	a = 0
	while a < 10:
		a += 1
		yield a
\end{lstlisting}
\end{solution}
\end{blocksection}


%%% Question %%%
\begin{blocksection}
\question Define \texttt{hailstone\_sequence} a generator that yields the hailstone sequence. Remember, for the hailstone sequence, if n is even, we need to divide by two, otherwise, we will multiply by 3 and add by 1. 

\begin{lstlisting}
; Doctests:
>>> hs_gen = hailstone_sequence(10)
>>> hs_gen.__next__()
10
>>> next(hs_gen) #equivalent to previous
5
>>> for i in hs_gen:
>>>	print(i)
16
8
4
2
1
\end{lstlisting}

\begin{solution}[1.5in]
\begin{lstlisting}
def hailstone_sequence(n):
	while n != 1:
		yield n
		if n % 2 == 0:
			n = n // 2
		else:
			n = n*3 + 1
	yield n
\end{lstlisting}
\end{solution}
\end{blocksection}

\question (Optional) Define \texttt{tree\_sequence} a generator that iterates through a tree by first yielding the root value and then yield each branch.
\begin{lstlisting}
>>> tree = Tree(1, [Tree(2, [Tree(5)]), Tree(3, [Tree(4)])])
>>> print(list(tree_sequence(tree)))
[1, 2, 5, 3, 4]
\end{lstlisting}

\begin{solution}[0.5in]
\begin{lstlisting}
def tree_sequence(tree):
	yield tree.entry
	for branch in tree.branches:
		for value in tree_sequence(branch):
			yield value
\end{lstlisting}
\end{solution}

%%% Question %%%
\section{Streams}
\begin{blocksection}
\question What’s the advantage of using a stream over a linked list?
\begin{solution}[0.5in] 
Lazy evaluation. We only evaluate up to what we need.
\end{solution}

\question What’s the maximum size of a stream?
\begin{solution}[0.5in]
Infinity
\end{solution}

\question What’s stored in first and rest? What are their types? 
\begin{solution}[0.5in]
First is a value, rest is another stream (either a method to calculate it, or an already calculated stream). In the case of Scheme, this is called a promise.
\end{solution}

\question When is the next element actually calculated?
\begin{solution}[.5in]
Only when it's requested (and hasn't already been calculated)
\end{solution}
\end{blocksection}


\section{What Would Scheme Print?}
\begin{blocksection}
\question For each of the following lines of code, write what scheme would output.

\begin{lstlisting}
scm> (define x 1)
\end{lstlisting}
\begin{solution}[.5in]
\texttt{x}
\end{solution}

\begin{lstlisting}
scm> (if 2 3 4)
\end{lstlisting}
\begin{solution}[.5in]
\texttt{3}
\end{solution}

\begin{lstlisting}
scm> (delay (+ x 1))
\end{lstlisting}
\begin{solution}[.5in]
\begin{lstlisting}
#[promise]
\end{lstlisting}
\end{solution}

\begin{lstlisting}
scm> (define (foo x) (+ x 10))
\end{lstlisting}
\begin{solution}[.5in]
\texttt{foo}
\end{solution}

\begin{lstlisting}
scm> (define bar (cons-stream (foo 1) (cons-stream (foo 2) bar)))
\end{lstlisting}
\begin{solution}[.5in]
\texttt{bar}
\end{solution}
\end{blocksection}

\begin{blocksection}
\begin{lstlisting}
scm> (car bar)
\end{lstlisting}
\begin{solution}[.5in]
\texttt{11}
\end{solution}

\begin{lstlisting}
scm> (cdr bar)
\end{lstlisting}
\begin{solution}[.5in]
\begin{lstlisting}
#[promise]
\end{lstlisting}
\end{solution}

\begin{lstlisting}
scm> (define (foo x) (+ x 1))
\end{lstlisting}
\begin{solution}[.5in]
\texttt{foo}
\end{solution}

\begin{lstlisting}
scm> (cdr-stream bar)
\end{lstlisting}
\begin{solution}[.5in]
\begin{lstlisting}
(3 . #[promise])
\end{lstlisting}
\end{solution}

\begin{lstlisting}
scm> (define (foo x) (+ x 5))
\end{lstlisting}
\begin{solution}[.5in]
\texttt{foo}
\end{solution}

\begin{lstlisting}
scm> (car bar)
\end{lstlisting}
\begin{solution}[.5in]
\texttt{11}
\end{solution}

\begin{lstlisting}
scm> (cdr-stream bar)
\end{lstlisting}
\begin{solution}[.5in]
\begin{lstlisting}
(3 . #[promise])
\end{lstlisting}
\end{solution}

\end{blocksection}

\section{Code Writing for Streams}

%%% Question %%%
\begin{blocksection}
\question Write out \texttt{double\_naturals}, which is a stream that evaluates to the sequence 1, 1, 2, 2, 3, 3, etc.
\begin{lstlisting}
(define (double_naturals)
    (double_naturals_helper 1 0)
)

(define (double_naturals_helper first flag)










)

\end{lstlisting}

\begin{solution}
\begin{lstlisting}
(define (double_naturals_helper first flag)
    (if (= 1 flag)
        (cons-stream first (double_naturals_helper (+ 1 first) 0))
        (cons-stream first (double_naturals_helper first 1))
    )
)

;Alternative Solutions
(define (double_naturals_helper first flag)
    (cons-stream first (double_naturals_helper (+ flag first) (- 1 flag)))
)
\end{lstlisting}
\end{solution}
\end{blocksection}

\begin{blocksection}
\question Write out \texttt{interleave}, which returns a stream that alternates between the values in stream1 and stream2. Assume that the streams are infinitely long.
\begin{lstlisting}
(define (interleave stream1 stream2)













)
\end{lstlisting}

\begin{solution}
\begin{lstlisting}
(define (interleave stream1 stream2)
(cons-stream (car stream1) 
 (interleave stream2 (cdr-stream stream1)))

(cons-stream (car stream1)
 (cons-stream (car stream2)
  (interleave (cdr-stream stream1)
  (cdr-stream stream2))))
)
\end{lstlisting}
\end{solution}
\end{blocksection}


\end{questions}

%%%%%%%%%%%%%%%%%%%%%%%%%%%%%%%%%%%%%%%%%%%%%%%%%%%%%%%%%%%%%%%%%%%%%%%%%%%%%%%

\end{document}
