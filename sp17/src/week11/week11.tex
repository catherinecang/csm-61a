%%%%%%%%%%%%%%%%%%%%%%%%%%%%%%%%%%%%%%%%%%%%%%%%%%%%%%%%%%%%%%%%%%%%%%%%%%%%%%%
%  _______  _______  ___      ____   _______
% |       ||       ||   |    |    | |   _   |
% |       ||  _____||   |___  |   | |  |_|  |
% |       || |_____ |    _  | |   | |       |
% |      _||_____  ||   | | | |   | |       |
% |     |_  _____| ||   |_| | |   | |   _   |
% |_______||_______||_______| |___| |__| |__|
%
% Header for CS61A Tutoring TeX Files
%
% Collected and modified by Jonathan Kotker and Thomas Magrino
%
% This package contains references to other packages, new commands, and values
% for different parameters, common to all discussion documents.
% This file should be placed one level above the discussion TeX files in
% the directory tree.
%
% Please modify the appropriate section below with information about
% the class: this information needs to be updated only once per semester.
%%%%%%%%%%%%%%%%%%%%%%%%%%%%%%%%%%%%%%%%%%%%%%%%%%%%%%%%%%%%%%%%%%%%%%%%%%%%%%%

%% Uncomment the following line to display solutions
%\def\discussionsolutions{}

\ProvidesPackage{commonheader}

%%% Packages Needed %%%%%%%%%%%%%%%%%%%%%%%%%%%%%%%%%%%%%%%%%%%%%%%%%%%%%%%%%%%
\usepackage{amsfonts}
\usepackage{amsmath}
\usepackage{bm}
\usepackage{amssymb}
\usepackage{amsthm}
\usepackage{color}
\usepackage{comment}
\usepackage{enumerate}
\usepackage{fancybox}
\usepackage{fix-cm}
\usepackage{float}
\usepackage{fullpage}
\usepackage{graphicx}
\usepackage[pdftex,colorlinks]{hyperref}
\usepackage{import}
\usepackage{listings}
\usepackage{multirow}
\usepackage{palatino}
\usepackage{paralist}
\usepackage{parskip}
\usepackage{tabularx}
\usepackage[calcwidth]{titlesec}
\usepackage{upquote}
\usepackage{verbatim}
\usepackage{wasysym}
\usepackage{tikz}
\usepackage{tikz-qtree,tikz-qtree-compat}
\usepackage{synttree}
%%%%%%%%%%%%%%%%%%%%%%%%%%%%%%%%%%%%%%%%%%%%%%%%%%%%%%%%%%%%%%%%%%%%%%%%%%%%%%%

%%% Information about the class %%%%%%%%%%%%%%%%%%%%%%%%%%%%%%%%%%%%%%%%%%%%%%%
\newcommand{\semester}[1]{\newcommand{\sem}{#1}}
\newcommand{\stafflist}[1]{\newcommand{\staff}{#1}}
%%%%%%%%%%%%%%%%%%%%%%%%%%%%%%%%%%%%%%%%%%%%%%%%%%%%%%%%%%%%%%%%%%%%%%%%%%%%%%%

%%% MODIFY THIS INFORMATION AS NECESSARY %%%%%%%%%%%%%%%%%%%%%%%%%%%%%%%%%%%%%%
\semester{Spring 2017}
\stafflist{Paul Bitutsky and Katya Stukalova, with \\
    Colin Schoen,
    Kevin Lin,
    Jason Goodman,
    Thomas Zhang,
    Danelle Nachum, 
    Natalia Layson,
    Jennie Chen,
    Nipun Ramakrishnan,
    Michael Gibbes,
    Christopher Allsman
}
%%%%%%%%%%%%%%%%%%%%%%%%%%%%%%%%%%%%%%%%%%%%%%%%%%%%%%%%%%%%%%%%%%%%%%%%%%%%%%%

%%% Margins and definitions %%%%%%%%%%%%%%%%%%%%%%%%%%%%%%%%%%%%%%%%%%%%%%%%%%%
\setlength{\marginparwidth}{1.2in}
\let\oldmarginpar\marginpar
\renewcommand{\marginpar}[1]{\-\oldmarginpar[\raggedleft\footnotesize #1]%
{\raggedright\footnotesize #1}}
%%%%%%%%%%%%%%%%%%%%%%%%%%%%%%%%%%%%%%%%%%%%%%%%%%%%%%%%%%%%%%%%%%%%%%%%%%%%%%%

%%% Configuring the document title %%%%%%%%%%%%%%%%%%%%%%%%%%%%%%%%%%%%%%%%%%%%
\definecolor{titlecol}{gray}{0.6}

\let\oldtitle\title
\renewcommand{\title}[1]
    {\oldtitle{
        {\begin{flushright}
            \huge {\sc{#1}}
            \fontsize{45}{0} \sffamily \color{titlecol} \selectfont
        \end{flushright}}
        \rule{\textwidth}{0.2em}}
    \newcommand{\disctitle}{#1}}
%%%%%%%%%%%%%%%%%%%%%%%%%%%%%%%%%%%%%%%%%%%%%%%%%%%%%%%%%%%%%%%%%%%%%%%%%%%%%%%

%%% Configuring the header and footer %%%%%%%%%%%%%%%%%%%%%%%%%%%%%%%%%%%%%%%%%
\newcommand{\discnumber}[1]{\newcommand{\discnum}{#1}}

\pagestyle{headandfoot}
\runningheadrule
\runningheader{\sc Group Tutoring handout \discnum: \disctitle}{}{Page \thepage}
\runningfootrule
\runningfooter{\small Computer Science Mentors CS61A \sem: \staff}{}{}
%%%%%%%%%%%%%%%%%%%%%%%%%%%%%%%%%%%%%%%%%%%%%%%%%%%%%%%%%%%%%%%%%%%%%%%%%%%%%%%

%%% Configuring the section and subsection titles %%%%%%%%%%%%%%%%%%%%%%%%%%%%%
\titleformat{\section}[hang]{\sffamily \bfseries}
{\fontsize{15}{25} \color{titlecol} \selectfont \thesection}{10pt}
{\fontfamily{ppl} \fontsize{15}{0} \selectfont \raggedleft}
[{\titlerule[1.0pt]}]

\titleformat{\subsection}[hang]{\sffamily \bfseries}
{\thesubsection}{10pt}{}[{\titlerule[0.5pt]}]
%%%%%%%%%%%%%%%%%%%%%%%%%%%%%%%%%%%%%%%%%%%%%%%%%%%%%%%%%%%%%%%%%%%%%%%%%%%%%%%

%%% Colored titles for sections %%%%%%%%%%%%%%%%%%%%%%%%%%%%%%%%%%%%%%%%%%%%%%%
\newcommand{\colorsec}[2]{\section[#1]{{\color{#2} #1}}}
\newcommand{\colorsubsec}[2]{\subsection[#1]{{\color{#2} #1}}}
\newcommand{\colorsubsubsec}[2]{\subsubsection[#1]{{\color{#2} #1}}}
%%%%%%%%%%%%%%%%%%%%%%%%%%%%%%%%%%%%%%%%%%%%%%%%%%%%%%%%%%%%%%%%%%%%%%%%%%%%%%%

%%% Commands used to add rules to tables and figures %%%%%%%%%%%%%%%%%%%%%%%%%%
\floatstyle{ruled}
\newfloat{ruledfigure}{tbph!}{lop}
\floatname{ruledfigure}{Figure}

\newfloat{ruledtable}{tbph!}{lop}
\floatname{ruledtable}{Table}
%%%%%%%%%%%%%%%%%%%%%%%%%%%%%%%%%%%%%%%%%%%%%%%%%%%%%%%%%%%%%%%%%%%%%%%%%%%%%%%

%%% Commands for references %%%%%%%%%%%%%%%%%%%%%%%%%%%%%%%%%%%%%%%%%%%%%%%%%%%
\renewcommand{\eqref}[1]{\hyperref[#1]{Equation \ref*{#1}}}
\newcommand{\exref}[1]{\hyperref[#1]{exercise \ref*{#1}}}
\newcommand{\figref}[1]{\hyperref[#1]{Figure \ref*{#1}}}
\newcommand{\lccderef}[1]{\hyperref[#1]{LCCDE \ref*{#1}}}
\newcommand{\quesref}[1]{\hyperref[#1]{question \ref*{#1}}}
\newcommand{\secref}[1]{\hyperref[#1]{section \ref*{#1}}}
\newcommand{\stepref}[1]{\hyperref[#1]{step \ref*{#1}}}
\newcommand{\tableref}[1]{\hyperref[#1]{Table \ref*{#1}}}
\renewcommand{\equationautorefname}{equation}
%%%%%%%%%%%%%%%%%%%%%%%%%%%%%%%%%%%%%%%%%%%%%%%%%%%%%%%%%%%%%%%%%%%%%%%%%%%%%%%

%%% Useful new commands %%%%%%%%%%%%%%%%%%%%%%%%%%%%%%%%%%%%%%%%%%%%%%%%%%%%%%%
\newcommand{\email}[1]{\href{#1}{{\tt #1}}}

\newcommand{\super}[1]{\ensuremath{^{\textrm{#1}}}}
\newcommand{\sub}[1]{\ensuremath{_{\textrm{#1}}}}

\newcommand{\boxtext}[2]{\framebox{\parbox[b]{#1}{#2}}}

% Defining terms
\newcommand{\define}[1]{\textbf{#1}}

% Rendering an unnumbered footnote
\makeatletter
\long\def\unmarkedfootnote#1{{\long\def\@makefntext##1{##1}\footnotetext{#1}}}
\makeatother
%%%%%%%%%%%%%%%%%%%%%%%%%%%%%%%%%%%%%%%%%%%%%%%%%%%%%%%%%%%%%%%%%%%%%%%%%%%%%%%

%%% Box sizes %%%%%%%%%%%%%%%%%%%%%%%%%%%%%%%%%%%%%%%%%%%%%%%%%%%%%%%%%%%%%%%%%
%%% These parameters determine the sizes of the boxes used %%%%%%%%%%%%%%%%%%%%
%%% to explain special and/or subtle points. %%%%%%%%%%%%%%%%%%%%%%%%%%%%%%%%%%
%%%%%%%%%%%%%%%%%%%%%%%%%%%%%%%%%%%%%%%%%%%%%%%%%%%%%%%%%%%%%%%%%%%%%%%%%%%%%%%
\newcommand{\boxlarge}{6.40in}
\newcommand{\boxmed}{6.05in}
\newcommand{\boxsmall}{5.75in}
%%%%%%%%%%%%%%%%%%%%%%%%%%%%%%%%%%%%%%%%%%%%%%%%%%%%%%%%%%%%%%%%%%%%%%%%%%%%%%%

%%% Miscellaneous parameters %%%%%%%%%%%%%%%%%%%%%%%%%%%%%%%%%%%%%%%%%%%%%%%%%%
\setlength{\parindent}{0in}
%\settimeformat{ampmtime}
\allowdisplaybreaks
%%%%%%%%%%%%%%%%%%%%%%%%%%%%%%%%%%%%%%%%%%%%%%%%%%%%%%%%%%%%%%%%%%%%%%%%%%%%%%%

%%% Mathematical commands %%%%%%%%%%%%%%%%%%%%%%%%%%%%%%%%%%%%%%%%%%%%%%%%%%%%%
\newcommand{\surjects}{\twoheadrightarrow}
\newcommand{\injects}{\hookrightarrow}
\newcommand{\isom}{\simeq}
\newcommand{\notdiv}{\nmid}
\newcommand{\del}{\partial}
\newcommand{\Intersection}{\bigcap} % intersection of a collection
\newcommand{\intersect}{\cap} % binary intersection
\newcommand{\Union}{\bigcup} % union of a collection
\newcommand{\union}{\cup} % binary union
\newcommand{\tensor}{\otimes}
\newcommand{\directsum}{\oplus} % binary direct sum
\newcommand{\Directsum}{\bigoplus} % direct sum of a collection
\newcommand{\isomto}{\overset{\sim}{\rightarrow}}

% Expected value
\newcommand{\E}[0]{\ensuremath{\mathbb{E}}}

% Probability (of an event)
\renewcommand{\P}[0]{\ensuremath{\mathbb{P}}}

% Variance/covariance
\newcommand{\var}[0]{\text{var}}
\newcommand{\cov}[0]{\text{cov}}

% Fraction w/parens around it
\newcommand{\pfrac}[2]{\ensuremath{\left(\frac{#1}{#2}\right)}}

% Fraction you can use even if you're not in math mode
\newcommand{\mfrac}[2]{\ensuremath{\frac{#1}{#2}}}

% "maybe equal" - equal with a ? sign on top
\newcommand{\meq}[0]{\ensuremath{\stackrel{?}{=}}}

% maybe anything - anything with a ? on top
\newcommand{\maybe}[1]{\ensuremath{\stackrel{?}{#1}}}

% \problem{1.1} gives you a spiffy-looking "Problem 1.1"
\newcommand{\problem}[1]{ \vspace{.15in} \noindent{\bf Problem #1} \quad\\
\noindent}

% negative/positive infinity
\newcommand{\ninfty}[0]{{\ensuremath{{-\infty}}}}
\newcommand{\pinfty}[0]{{\ensuremath{{+\infty}}}}

% Derivative wrt argument
\newcommand{\der}[1]{\ensuremath{\frac{d}{d #1}}}

% Partial derivative of arg1 wrt arg2
\newcommand{\pder}[2]{\ensuremath{\frac{\partial #1}{\partial #2}}}

% nth derivative
\newcommand{\nder}[2]{\ensuremath{\frac{d^{#2}}{d #1^{#2}}}}

% floor and ceiling
\newcommand{\floor}[1]{\ensuremath{\lfloor{} #1 \rfloor{}}}
\newcommand{\ceil}[1]{\ensuremath{\lceil{} #1 \rceil{}}}

% like \boxed{} but for text instead of math
\newcommand{\tboxed}[1]{\boxed{\text{#1}}}

\def\({\left(}
\def\){\right)}

\def\<{\left\langle}
\def\>{\right\rangle}

% For lemmas
\newtheorem{lemma}{Lemma}

% For Python code
\lstset{language=Python,
        basicstyle=\ttfamily,
        commentstyle=\ttfamily,
        showstringspaces=false,
        breaklines=true,
}

% For Scheme code
\lstdefinelanguage{Scheme}{
  morekeywords=[1]{define, define-syntax, define-macro, lambda, define-stream,
                   stream-lambda},
  morekeywords=[2]{begin, call-with-current-continuation, call/cc,
                   call-with-input-file, call-with-output-file, case, cond, do,
                   else, for-each, if, let*, let, let-syntax, letrec,
                   letrec-syntax, let-values, let*-values, and, or, not, delay,
                   force, quasiquote, quote, unquote, unquote-splicing, map,
                   fold, syntax, syntax-rules, eval, environment, query},
  morekeywords=[3]{import, export},
  alsodigit=!\$\%&*+-./:<=>?@^_~,
  sensitive=true,
  morecomment=[l]{;},
  morecomment=[s]{\#|}{|\#},
  morestring=[b]",
  basicstyle=\ttfamily,
  keywordstyle=\bf\ttfamily,
  upquote=true,
  breaklines=true,
  breakatwhitespace=true,
  literate=*{`}{{`}}{1}
}

% Remove spacing before and after `lstlisting` block, so there's no extra space
% between two blocks.
\lstset{aboveskip=0pt,belowskip=0pt}

% Creates a minipage environment, which can be used to section off text so that
% it won't be split between two pages. Minipage sets the parskip variable to 0,
% so this preserves the value in the minipage environment.
%
% Source:
% http://tex.stackexchange.com/questions/43002/how-to-preserve-the-same-parskip-in-minipage
\newlength{\currentparskip}
\newenvironment{blocksection}
{
    \setlength{\currentparskip}{\parskip}% save the value
    \begin{minipage}{\linewidth}
    \setlength{\parskip}{\currentparskip}% restore the value
}
{
    \end{minipage}
}
    

%%%%%%%%%%%%%%%%%%%%%%%%%%%%%%%%%%%%%%%%%%%%%%%%%%%%%%%%%%%%%%%%%%%%%%%%%%%%%%%

% Use \printanswers to print with answers and \noprintanswers to print without.
% Check for defined flag \discussionsolutions to show solutions
\ifdefined\discussionsolutions
  \printanswers
\else
  \noprintanswers
\fi

% The nonsol environment will show items within it only if answers are not set
% to be printed. Useful for hiding docstring in worksheet solutions.
\newenvironment{nonsol}{}{}
\ifdefined\discussionsolutions
    \excludecomment{nonsol}
\else
\fi
%%%%%%%%%%%%%%%%%%%%%%%%%%%%%%%%%%%%%%%%%%%%%%%%%%%%%%%%%%%%%%%%%%%%%%%%%%%%%%%

\author{\Large \sc Computer Science Mentors 61A}

