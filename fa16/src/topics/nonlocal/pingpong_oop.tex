\question \textbf{(Optional)} Instead of using nonlocal for pingpong, let's use OOP!

\begin{nonsol}
\begin{lstlisting}
>>> tracker1 = PingPongTracker()
>>> tracker2 = PingPongTracker()
>>> tracker1.next()
1
>>> tracker1.next()
2
>>> tracker2.next()
1

class PingPongTracker:
    def __init__(self):
        self.current = 0
        self.index = 1
        self.add = True
    def next(self):
        """*** Enter solution below ***"""
\end{lstlisting}
\end{nonsol}

\begin{solution}[0.3in]
\begin{lstlisting}
class PingPongTracker:
    def __init__(self):
        self.current = 0
        self.index = 1
        self.add = True

    def next(self):
        if self.add:
            self.current += 1
        else:
            self.current -= 1
        if has_seven(self.index) or self.index % 7 == 0:
            self.add = not self.add
        self.index += 1
        return self.current


\end{lstlisting}
Notice how the OOP approach is very similar to the nonlocal solution.
Instead of using \texttt{nonlocal}, we use \texttt{self.variable} and the code
becomes exactly the same. We just store the data in a slightly different way.
This implies that OOP and functions are pretty similar, and it turns out you can
even write your own OOP framework using just functions and \texttt{nonlocal}!

In addition, there are a lot of Python specific features that can be written
using functions or using classes. If you are interested, check out the powerful
Python feature decorators, and note how we can write them both as functions and
as classes!
\end{solution}
