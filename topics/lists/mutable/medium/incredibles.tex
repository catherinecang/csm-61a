\begin{blocksection}
\question Draw the box-and-pointer diagram. \\

\begin{lstlisting}
>>> violet = [7, 77, 17]
>>> violet.append([violet.pop(1)])

>>> dash = violet * 2
>>> jack = dash[3:5]
>>> jackjack = jack.extend(jack)

>>> helen = list(violet)
>>> helen += [jackjack]
>>> helen[2].append(violet)
\end{lstlisting}

\begin{solution}[1in]
\url{https://goo.gl/EAmZBW}

Explanation:
1) Remember that pop is the mutation function that returns something. When appending a list,
remember to make it a nested list.
2) When copying a list, remember that it is a "shallow copy". This means we  want to copy
whatever is in that box in the box and pointer diagram, so if there's a value there, we copy the value. However,
if there is a nested list, we copy the arrow pointing to the list. We do not create a new list.
3) All mutation functions except \textbf{pop} do not have returns.
\end{solution}
\end{blocksection}
