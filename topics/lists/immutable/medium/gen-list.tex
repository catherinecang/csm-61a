\begin{blocksection}
\question Fill in the methods below according to the doctests.
\end{blocksection}

\begin{lstlisting}
def gen_list(n):
    """
    Returns a nested list structure of n elements where the ith
    element is a list from 0 (inclusive) to i (exclusive).
    >>> gen_list(3)
    [[0], [0, 1], [0, 1, 2]]
    >>> gen_list(5)
    [[0], [0, 1], [0, 1, 2], [0, 1, 2, 3], [0, 1, 2, 3, 4]]
    """
    return _______________________________________________
\end{lstlisting}

For an additional challenge, try out the following:

\begin{lstlisting}
def gen_increasing(n):
    """
    Returns a nested list structure of n elements where the the element of each 
    list is one more than the previous element (even if the previous is in a prior sublist).
    >>> gen_increasing(3)
    [[0], [1, 2], [3, 4, 5]]
    >>> gen_list(5)
    [[0], [1, 2], [3, 4, 5], [6, 7, 8, 9], [10, 11, 12, 13, 14]]
    """
    return ______________________________________________
\end{lstlisting}

\textbf{Hint:} You can sum ranges. E.g. \texttt{sum(range(3))} gives us $0 + 1 + 2 = 3$.

\begin{solution}
\begin{lstlisting}
def gen_list(n):
    return [[i for i in range(j+1)] for j in range(n)]

def gen_increasing(n):
	return [[i for i in range(sum(range(j+1)), sum(range(j+1)) + j+1)] for j in range(n)]
\end{lstlisting}
\end{solution}