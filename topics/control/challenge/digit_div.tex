\question Fill out the function \lstinline{digit_div} which returns an integer that contains in any order all the digits of \lstinline{k} that divide \lstinline{n} evenly. If no such digit of \lstinline{k} exists, the function should return 0. Assume that both \lstinline{n} and \lstinline{k} are positive integers.

\begin{blocksection}
\begin{lstlisting}
def digit_div (n, k):
    >>> digit_div(4, 1234567890)
    421
    >>> digit_div(4, 2323)
    22
    >>> digit_div(7, 2323)
    0
    """
    ______________________
    while _________________________:
        curr_digit = k % 10
        if _________________________________________:
           _________________________________________
        ____________________________________________
    return _______________
\end{lstlisting}
\end{blocksection}

\begin{blocksection}
\begin{solution}[0in]
\begin{lstlisting}
def digit_div (n, k):
    digits = 0
    while k > 0:
        curr_digit = k % 10
        if curr_digit != 0 and n % curr_digit == 0:
            digits = digits * 10 + curr_digit
        k //= 10
    return digits
\end{lstlisting}
Notice that in the condition for the \lstinline{if} statement, we must check if \lstinline{curr_digit} is 0 before we divide \lstinline{n} by \lstinline{curr_digit} otherwise, we may try to divide \lstinline{n} by 0 which would error! This line works because of short circuiting; if the first expression in an \lstinline{and} expression is a false value, then the second expression is never evaluated.
\end{solution}
\end{blocksection}