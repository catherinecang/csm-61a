\begin{blocksection}
\question Complete the implementation of \lstinline{iter_link}, which takes in a linked list and returns a generator which will iterate over the values of the linked list in order. Your function should support deep linked lists.

\begin{lstlisting}
def iter_link(lnk):
  """ 
  Yield the values of a linked list in order; your function should support deep linked lists.
  >>> lst1 = Link(1, Link(2, Link(3, Link(4))))
  >>> list(iter_link(lst1))
  [1, 2, 3, 4]
  >>> lst2 = Link(1, Link(Link(2, Link(3)), Link(4, Link(5))))
  >>> print(lst2)
  <1 <2 3> 4 5>
  >>> iter_lst2 = iter_link(lst2)
  >>> next(iter_lst2)
  1
  >>> next(iter_lst2)
  2
  >>> next(iter_lst2) 
  3
  >>> next(iter_lst2)
  4
  """
  if lnk is not Link.empty:
    if type(___________) is Link:
      ___________________________
    else:
      ___________________________
    _____________________________
\end{lstlisting}

\begin{solution}[0.7in]
\begin{lstlisting}
def iter_link(lnk):
  if lnk is not Link.empty:
    if type(lnk.first) is Link:
      yield from iter_link(lnk.first)
    else:
      yield lnk.first
    yield from iter_link(lnk.rest)
\end{lstlisting}
\end{solution}
\end{blocksection}
