\begin{blocksection}
\question Write a function \lstinline$is_sorted$ that takes in an integer
\lstinline$n$ and returns true if the digits of that number are nondecreasing from
right to left.

\begin{lstlisting}
def is_sorted(n):
   """
   >>> is_sorted(2)
   True
   >>> is_sorted(22222)
   True
   >>> is_sorted(9876543210)
   True
   >>> is_sorted(9087654321)
   False
   """
\end{lstlisting}
\end{blocksection}


\begin{blocksection}
\begin{solution}[1in]
\begin{lstlisting}
    right_digit = n % 10
    rest =  n // 10
    if rest == 0:
        return True
    elif right_digit > rest % 10:
        return False
    else:
        return is_sorted(rest)
\end{lstlisting}

First, let’s look into the base case. At what point will you know a number is sorted/not sorted immediately? 1) if \lstinline{n} only has 1 digit or is 0, we know it is definitely sorted with itself. This corresponds to the first if condition, \lstinline{rest == 0}. 2)If the 2nd-to-last and last digits are not in sorted order, we know the number is not sorted. To do this, we need at least 2 digits in \lstinline{n} to compare, which is why we check this in elif after ensuring \lstinline{n} is not 0. 


Next, let’s go into the recursive step. We build off of the base cases: if the base cases fail, then we can now work off of the assumption that \lstinline{n} has at least 2 digits and the last 2 digits of n are in sorted order. Next, notice that after chopping off the last digit, to check that the rest of \lstinline{n} is sorted, we can use our function \lstinline{is_sorted} on the number \lstinline{rest}. So finally, we make the recursive call with \lstinline{rest} as the argument.

\end{solution}
\end{blocksection}
