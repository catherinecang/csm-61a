\begin{blocksection}
\question Consider the \lstinline$word_finder$ function below.

\begin{lstlisting}
def word_finder(t, n, word):
    if root(t) == word:
        n -= 1
        if n == 0:
            return True
    for branch in branches(t):
        if word_finder(branch, n, word):
            return True
    return False
\end{lstlisting}
\end{blocksection}

\begin{parts}
\part What does this function do?

\begin{solution}[1em]
This function takes a tree \lstinline$t$, an integer \lstinline$n$, and a string \lstinline$word$ in as input.

\lstinline$word_finder$ returns true if any paths from the root towards the leaves have at least \lstinline$n$ occurrences of the word and false otherwise.
\end{solution}

\part If a tree has $n$ total nodes, what is the total runtime for all searches
in $\Theta(\cdot)$ notation?

\begin{solution}[1em]
$\Theta(n)$. At worst, we must visit every node of the tree.
\end{solution}

\end{parts}
