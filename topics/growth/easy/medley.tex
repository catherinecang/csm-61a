\question What is the order of growth in time for the following functions? Use
big-$\Theta$ notation.
\begin{parts}

\begin{comment}
\part
\begin{lstlisting}
def strange_add(n):
    if n == 0:
        return 1
    else:
        return strange_add(n - 1) + strange_add(n - 1)
\end{lstlisting}
\begin{solution}[0.25in]
$\Theta(2^n)$. To see this, try drawing out the call tree. Each level will create
two new calls to \texttt{strange\char`_add}, and there are $n$ levels.
Therefore, $2^n$ calls.
\end{solution}
\end{comment}

\begin{comment}
\part
\begin{lstlisting}
def stranger_add(n):
    if n < 3:
        return n
    elif n % 3 ==  0:
        return stranger_add(n - 1) + stranger_add(n - 2) + stranger_add(n - 3)
    else:
        return n
\end{lstlisting}
\begin{solution}[0.25in]
$\Theta(n)$ is $n$ is a multiple of 3, otherwise $\Theta(1)$.\\
The case where $n$ is not a multiple of 3 is fairly obvious -- we step into the
else clause and immediately return.\\
If $n$ is a multiple of 3, then neither $n - 1$ nor $n - 2$ are multiples of 3
so those calls will take constant time. Therefore, we just run
\texttt{stranger\char`_add}, decrementing the argument by 3 each time.
\end{solution}
\end{comment}

\begin{comment}
\part
\begin{lstlisting}
def waffle(n):
    i = 0
    total = 0
    while i < n:
        for j in range(50 * n):
            total += 1
        i += 1
    return total
\end{lstlisting}
\begin{solution}[0.25in]
$\Theta(n^2)$. Ignore the constant term in $50 * n$, and it because just two for
loops.
\end{solution}
\end{comment}

\part
\begin{lstlisting}
def belgian_waffle(n):
    i = 0
    total = 0
    while i < n:
        for j in range(50 * n ** 2):
            total += 1
        i += 1
    return total
\end{lstlisting}
\begin{solution}[0.25in]
$\Theta(n^3)$. Inner loop runs $n^2$ times, and the outer loop runs $n$ times. To get
the total, multiply those together.
\end{solution}

\part
\begin{lstlisting}
def pancake(n):
    if n == 0 or n == 1:
        return n
    # Flip will always perform three operations and return -n.
    return flip(n) + pancake(n - 1) + pancake(n - 2)
\end{lstlisting}
\begin{solution}[0.25in]
$\Theta(2^n)$. Flip will run in constant time. Therefore, this call tree looks very
similar to fib! (which is $2^n$)
\end{solution}

\part
\begin{lstlisting}
def toast(n):
    i, j, stack = 0, 0, 0
    while i < n:
        stack += pancake(n)
        i += 1
    while j < n:
        stack += 1
        j += 1
    return stack
\end{lstlisting}
\begin{solution}[0.25in]
$\Theta(n 2^n)$. There are two loops: the first runs $n$ times for $2^n$ calls each
time (due to pancake), for a total of $n 2^n$. The second loop runs $n$ times.
When calculating orders of growth however, we focus on the dominating term -- in
this case, $n 2^n$.
\end{solution}


\end{parts}

