Unlike Python, all Scheme lists are linked lists. Recall that a linked list is made up of Links that have a first and a rest, where the rest is another Link. Similarly, Scheme lists are made up of pairs with a first and a rest, where the rest is another pair.

\textbf{Ways to make scheme lists:}
\begin{itemize}
\item Cons \\
\textbf{Syntax:} \lstinline{(cons <car-elem> <cdr-elem>)} \\
Takes in a pair of two elements; similar to how a Python linked list has 2 elements as well --- first and rest.
\item List \\
\textbf{Syntax:} \lstinline{(list <elem1> <elem2> ...)} \\
Takes in an arbitrary number of elements/arguments, and constructs a list where each elem is the first of its own pair. Note how this differs from \lstinline{cons} where you specify a first and rest rather than just specifying the first of each pair. All the arguments will be evaluated before being collected into the scheme list.
\item ' (aka single quote) \\
\textbf{Syntax:} \lstinline{'(<elem1> <elem2> ...)} \\
Also takes in an arbitrary number of elements and construct a list out of the elements, but the arguments are not evaluated.
\end{itemize}

\textbf{Ways to access list items:}
\begin{itemize}
\item car \\
\textbf{Syntax:} \lstinline{(car <pair>)} \\
Gets you the first item of a pair
\item cdr \\
\textbf{Syntax:} \lstinline{(cdr <pair>)} \\
Gets you the second item of a pair
\end{itemize}
\begin{itemize}
\item cadr \\
\textbf{Syntax:} \lstinline{(cadr <pair>)} \\
Gets you the \lstinline{car} of the \lstinline{cdr}
\end{itemize}
\begin{itemize}
\item cddr \\
\textbf{Syntax:} \lstinline{(cddr <pair>)} \\
Gets you the \lstinline{cdr} of the \lstinline{cdr}
\end{itemize}
