\begin{comment}
\begin{blocksection}
\question Implement \texttt{waldo}. \texttt{waldo} returns \texttt{\#t} if the
symbol waldo is in a list. You may assume that the list passed in is
well-formed.\\

\begin{lstlisting}
scm> (waldo '(1 4 waldo))
#t
scm> (waldo '())
#f
scm> (waldo '(1 4 9))
#f

(define (waldo lst)














)
\end{lstlisting}

\begin{solution}
\begin{lstlisting}
(define (waldo lst)
    (cond ((null? lst) #f)
          ((eq? (car lst) 'waldo) #t)
          (else (waldo (cdr lst)))
      )
  )
\end{lstlisting}
\end{solution}
\end{blocksection}
\end{comment}

\begin{blocksection}
\question Define \texttt{waldo} which takes in a list. If that list contains the
symbol waldo, it returns the index where waldo first appears. Otherwise, it
returns \texttt{\#f}.
\begin{lstlisting}
scm> (waldo '(1 4 waldo))
2
scm> (waldo '())
#f
scm> (waldo '(1 4 9))
#f

(define (waldo lst)















)
\end{lstlisting}

\begin{solution}[0.5in]
\begin{lstlisting}
(define (waldo lst)
    (define (helper lst index)
        (cond ((null? lst) #f)
              ((eq? (car lst) 'waldo) index)
              (else (helper (cdr lst) (+ index 1)))))
    (helper lst 0))
\end{lstlisting}
\end{solution}
\end{blocksection}