OOP is a way of organizing programs so that we use objects to communicate statuses with each other about the state of the program. As an example:
class Car:
    wheels = 4
    def __init__(self):
        self.gas = 100

    def drive(self):
        if self.gas <= 0:
      print(“Can’t drive, need to go to the gas station!”)
        else:
	      self.gas -= 10
		
What is self? The self you see in the arguments in all the methods/dot notation refers to the current instance of a class. Dot notation with an instance before the dot automatically supplies the first argument to a method, so we don’t have to explicitly pass in a parameter for the self argument.
For example, there’s two ways of calling a method. These do the same thing:
my_car = Car()
Car.drive(my_car) # explicitly passes in argument to self
my_car.drive() # implicitly supplies value of self with dot notation
The method __init__ of a class, which we call the constructor, is called to create a new instance of that class.
In our code above, Car() makes a new instance of the Car class by calling the __init__ method of the car class. 
Again, if the __init__ method takes in only the self argument, nothing needs to be passed in to the constructor.
Class attributes vs. instance attributes
In the example above, the attribute wheels is shared by all instances of the Car class; while gas is an instance attribute that’s specific to the instance my_car
my_car.wheels and Car.wheels both return the value 4 in this case
Order of looking up an attribute: instance → class → parent class (if there is one)
Stop looking after finding the attribute
