\documentclass{exam}
\usepackage{../commonheader}
\lstset{language=Scheme}

%%% CHANGE THESE %%%%%%%%%%%%%%%%%%%%%%%%%%%%%%%%%%%%%%%%%%%%%%%%%%%%%%%%%%%%%%
\discnumber{13}
\title{\textsc{SQL Aggregation and Final Review}}
\date{August 12, 2019}
%%%%%%%%%%%%%%%%%%%%%%%%%%%%%%%%%%%%%%%%%%%%%%%%%%%%%%%%%%%%%%%%%%%%%%%%%%%%%%%

\begin{document}
\maketitle
\rule{\textwidth}{0.15em}
\fontsize{12}{15}\selectfont


\section{SQL Aggregation}
\subimport{../../topics/sql/fish/text/}{intro.tex}
\subimport{../../topics/sql/fish/tables/}{fish.tex}

Hint: The aggregate functions \texttt{MAX}, \texttt{MIN}, \texttt{COUNT}, and \texttt{SUM} return the maximum, minimum, number, and sum of the values in a column. The  \texttt{GROUP BY} clause of a select statement is used to partition rows into groups.

\newpage
\begin{questions}
\subimport{../../topics/sql/fish/easy/}{most-populated-species.tex}
\subimport{../../topics/sql/fish/easy/}{number-fish.tex}
\subimport{../../topics/sql/fish/easy/}{most-pieces-per-price.tex}
\end{questions}

%\section{Environment Diagrams}
%\begin{questions}
%\subimport{../../topics/lists/mutable/medium/env-diagram/}{env-prog.tex}
%\end{questions}

\newpage
\section{Recursive Data Structures}
\begin{questions}
\subimport{../../topics/interview/}{rot.tex}

\newpage
\subimport{../../topics/linked-lists/class/medium/}{shuffle.tex}
\end{questions}

\newpage
\section{Recursion}
\begin{questions}
\subimport{../../topics/interview/}{game.tex}
\end{questions}

\newpage
\section{Scheme}
\begin{questions}
\subimport{../../topics/scheme/medium/}{insert.tex}
\end{questions}


\newpage
\section{Iterators, Generators, and Streams}

\begin{questions}
\subimport{../../topics/interview/}{all-ways-skeleton.tex}
\newpage
\subimport{../../topics/streams/scheme/medium/}{puns.tex}
\end{questions}


\end{document}
