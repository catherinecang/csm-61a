\documentclass{exam}
\usepackage{../../commonheader}

%%% CHANGE THESE %%%%%%%%%%%%%%%%%%%%%%%%%%%%%%%%%%%%%%%%%%%%%%%%%%%%%%%%%%%%%%
\discnumber{5}
\title{\textsc{Orders of Growth}}
\date{March 7 to March 11, 2016}
%%%%%%%%%%%%%%%%%%%%%%%%%%%%%%%%%%%%%%%%%%%%%%%%%%%%%%%%%%%%%%%%%%%%%%%%%%%%%%%

\begin{document}
\maketitle
\rule{\textwidth}{0.15em}
\fontsize{12}{15}\selectfont

%%% INCLUDE TOPICS HERE %%%%%%%%%%%%%%%%%%%%%%%%%%%%%%%%%%%%%%%%%%%%%%%%%%%%%%%


%%% Question %%%

\begin{questions}
\begin{blocksection}
\question
In big-O notation, what is the runtime for \texttt{foo}?
\begin{parts}
\part
\begin{lstlisting}
def foo(n):
    for i in range(n):
        print('hello')
\end{lstlisting}
\begin{solution}[0.25in]
$O(n)$. This is simple loop that will run $n$ times.
\end{solution}

\part What's the runtime of \texttt{foo} if we change \texttt{range(n)}:
\begin{subparts}

\subpart To \texttt{range(n / 2)}?
\begin{solution}[0in]
$O(n)$. The loop runs $n / 2$ times, but we ignore constant factors.
\end{solution}

\subpart To \texttt{range(10)}?
\begin{solution}[0in]
$O(1)$. No matter the size of $n$, we will run the loop the same number of
times.
\end{solution}

\subpart To \texttt{range(10000000)}?
\begin{solution}[0in]
$O(1)$. No matter the size of $n$, we will run the loop the same number of
times.
\end{solution}

\end{subparts}
\end{parts}
\end{blocksection}

\question What is the order of growth in time for the following functions? Use
big-O notation.
\begin{parts}

\part
\begin{lstlisting}
def strange_add(n):
    if n == 0:
        return 1
    else:
        return strange_add(n - 1) + strange_add(n - 1)
\end{lstlisting}
\begin{solution}[0.25in]
$O(2^n)$. To see this, try drawing out the call tree. Each level will create
two new calls to \texttt{strange\char`_add}, and there are $n$ levels.
Therefore, $2^n$ calls.
\end{solution}

\part
\begin{lstlisting}
def stranger_add(n):
    if n < 3:
        return n
    elif n % 3 ==  0:
        return stranger_add(n - 1) + stranger_add(n - 2) + stranger_add(n - 3)
    else:
        return n
\end{lstlisting}
\begin{solution}[0.25in]
$O(n)$ is $n$ is a multiple of 3, otherwise $O(1)$.\\
The case where $n$ is not a multiple of 3 is fairly obvious -- we step into the
else clause and immediately return.\\
If $n$ is a multiple of 3, then neither $n - 1$ nor $n - 2$ are multiples of 3
so those calls will take constant time. Therefore, we just run
\texttt{stranger\char`_add}, decrementing the argument by 3 each time.
\end{solution}

\newpage
\part
\begin{lstlisting}
def waffle(n):
    i = 0
    sum = 0
    while i < n:
        for j in range(50 * n):
            sum += 1
        i += 1
    return sum
\end{lstlisting}
\begin{solution}[0.25in]
$O(n^2)$. Ignore the constant term in $50 * n$, and it because just two for
loops.
\end{solution}

\part
\begin{lstlisting}
def belgian_waffle(n):
    i = 0
    sum = 0
    while i < n:
        for j in range(n ** 2):
            sum += 1
        i += 1
    return sum
\end{lstlisting}
\begin{solution}[0.25in]
$O(n^3)$. Inner loop runs $n^2$ times, and the outer loop runs $n$ times. To get
the total, multiply those together.
\end{solution}

\part
\begin{lstlisting}
def pancake(n):
    if n == 0 or n == 1:
        return n
    # Flip will always perform three operations and return -n.
    return flip(n) + pancake(n - 1) + pancake(n - 2)
\end{lstlisting}
\begin{solution}[0.25in]
$O(2^n)$. Flip will run in constant time. Therefore, this call tree looks very
similar to fib! (which is $2^n$)
\end{solution}

\part
\begin{lstlisting}
def toast(n):
    i = 0
    j = 0
    stack = 0
    while i < n:
        stack += pancake(n)
        i += 1
    while j < n:
        stack += 1
        j += 1
    return stack
\end{lstlisting}
\begin{solution}[0.25in]
$O(n 2^n)$. There are two loops: the first runs $n$ times for $2^n$ calls each
time (due to pancake), for a total of $n 2^n$. The second loop runs $n$ times.
When calculating orders of growth however, we focus on the dominating term -- in
this case, $n 2^n$.
\end{solution}

\end{parts}

\begin{blocksection}
\question Consider the following functions: 
\begin{lstlisting}
def hailstone(n):
    print(n)
    if n < 2:
        return
    if n % 2 == 0:
        hailstone(n // 2)
    else:
        hailstone((n * 3) + 1)

def fib(n):
   if n < 2:
      return n
   return fib(n - 1) + fib(n - 2)

def foo(n, f):
    return n + f(500)
\end{lstlisting}
In big-O notation, describe the runtime for the following:
\begin{parts}
\part \texttt{foo(10, hailstone)}
\begin{solution}[0in]
$O(1)$. $f(500)$ is independent of the size of the input $n$.
\end{solution}
\part \texttt{foo(3000, fib)}
\begin{solution}[0in]
$O(1)$. See above.
\end{solution}
\end{parts}

\end{blocksection}

\begin{blocksection}
\question \textbf{Fast Exponentiation:} in this problem, we will examine a
real-world algorithm used to improve the speed of calculating exponents.

\begin{parts}

\part First, express the runtime of the naive exponentiation algorithm in big-O
notation.
\begin{lstlisting}
def exp(b, n):
    if n == 0:
        return 1
    else:
        return b * exp(b, n - 1)
\end{lstlisting}
\begin{solution}[0.25in]
$O(n)$. $n$ decreases by 1 each call, so there are naturally $n$ calls.
\end{solution}

\part Now, express the runtime of the fast exponentiation algorithm in big-O
notation.
\begin{lstlisting}
def fast_exp(b, n):
    if n == 0:
        return 1
    elif n % 2 == 0: # Assume square runs in constant time
        return square(fast_exp(b, n // 2))
    else:
        return b * fast_exp(b, n - 1)
\end{lstlisting}
\begin{solution}[0.25in]
$O(\log n)$. $n$ is halved each call, so the number of calls is the number of
times $n$ must be halved to get to 1. This is $\log n$.
\end{solution}

\part What about this slightly modified version of \texttt{fast\char`_exp}?
\begin{lstlisting}
def fast_exp(b, n):
    for _ in range(50 * n):
        print("Killing time")
    if n == 0:
        return 1
    elif n % 2 == 0:
        return square(fast_exp(b, n // 2))
    else:
        return b * fast_exp(b, n - 1)
\end{lstlisting}
\begin{solution}[0.25in]
$O(n)$. Ignore the constant term. The first call will perform $n$ operations,
the second call will perform $n / 2$ operations, the third will perform $n / 4$
operations, etc. Using geometric series, we see this adds up to $2n$, which is
$n$ if we ignore constant terms.
\end{solution}

\end{parts}
\end{blocksection}

\begin{blocksection}
\question \textbf{Mysterious loops:} What is the order of growth in time for the following functions? Use big-O notation.

\begin{parts}

\part
\begin{lstlisting}
def mystery(n):
    for i in range(n):
        while i % 2 != 0:
            print(i)
            i = i - 1
        print("Done")
\end{lstlisting}
\begin{solution}[0.25in]
$O(n)$. The work for when $i$ is divisible by two is constant. Subtracting one
will immediately allow us to exit the \texttt{while} loop. Therefore, we can
concentrate on just the outer loop.
\end{solution}

\part
\begin{lstlisting}
def fun(n):
    for i in range(n):
        for j in range(n * n):
            if j == 4:
                return -1
            print("Fun!")
\end{lstlisting}
\begin{solution}[0.25in]
$O(1)$. Inner loop always immediately exits after running for 4 iterations,
independent of $n$.
\end{solution}

\end{parts}
\end{blocksection}

\begin{blocksection}
\question \textbf{Orders of Growth and Trees:} Assume we are using the non-mutable Tree implementation introduced in discussion. Consider the following function:
\begin{lstlisting}
def word_finder(t, n, word):
    if root(t) == word:
        n -= 1
        if n == 0:
            return True
    for branch in branches(t):
        if word_finder(branch, n, word):
            return True
    return False
\end{lstlisting}

\begin{parts}
\part What does this function do?  \newline
\begin{solution}[0.25in]
This function take a Tree \texttt{t}, an integer \texttt{n}, and a string
\texttt{word} in as input.

Then, \texttt{word\char`_finder} returns True if any paths from the root towards
the leaves have at least \texttt{n} occurrences of the word and False otherwise.
\end{solution}

\part If a tree has $n$ total nodes, what is the total runtime for all searches
in big-O notation?
\begin{solution}[0.25in]
$O(n)$. At worst, we must visit every node of the tree.
\end{solution}

\end{parts}

\end{blocksection}

\begin{blocksection}
\question \textbf{Orders of Growth and Linked Lists:} Consider the following
linked list function:
\begin{lstlisting}
def insert_at_beginning(lst, x):
    return Link(x, lst)
\end{lstlisting}

\begin{parts}
\part What does this function do?

\begin{solution}
It takes in an existing lst and returns a new list with $x$ at the front.
\end{solution}

\part Assume lst is initially length $n$. How long does it take to do one
insert? Two? $n$?
\begin{solution}
All inserts will take constant time. No matter how long the list is, it doesn't
take any longer to add to the front. One insert will take one unit of time, and
two will take roughly twice that. Therefore, the amount of time to do $n$
inserts will be $O(n)$.
\end{solution}

Now consider:
\begin{lstlisting}
def insert_at_end(lst, x):
    if lst.rest is Link.empty:
        lst.rest = Link(x)
    else:
        insert_at_end(lst.rest, x)
\end{lstlisting}

\part What does this function do?
\begin{solution}[0.25in]
    Inserts a value \texttt{x} at the end of linked list \texttt{lst}.
\end{solution}
\part Say we want to repeatedly insert some numbers into the end of a linked
list:
\begin{lstlisting}
def insert_many_end(lst, n):
    for i in range(n):
        insert_at_end(lst, i)
\end{lstlisting}
\begin{subparts}

\subpart Assume \texttt{lst} is initially length 1. How long will it take to do
the first insertion? The second? The $n$th?
\begin{solution}[0.25in]
Notice that the list gets longer with each insertion, so each operation will
make it harder to do the next.
Therefore, the first insertion will take about 1 unit of time. The second will
take about twice as long, at two units of time. The $n$th insertion will take
$n$ units of time.
\end{solution}

\subpart In big-O notation, What is the total runtime to do all the inserts?
(total runtime of \texttt{insert\char`_many\char`_end})
\begin{solution}[0.25in]
The total runtime will be the sum of all the inserts: 1 + 2 + 3 + \ldots + $n$ =
$\frac{n (n + 1)}{2}$ = $O(n^{2})$
\end{solution}

\end{subparts}

\end{parts}
\end{blocksection}

\end{questions}

%%%%%%%%%%%%%%%%%%%%%%%%%%%%%%%%%%%%%%%%%%%%%%%%%%%%%%%%%%%%%%%%%%%%%%%%%%%%%%%

\end{document}
