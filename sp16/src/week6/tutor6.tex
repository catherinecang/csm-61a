\documentclass{exam}
\usepackage{../../commonheader}

%%% CHANGE THESE %%%%%%%%%%%%%%%%%%%%%%%%%%%%%%%%%%%%%%%%%%%%%%%%%%%%%%%%%%%%%%
\discnumber{3}
\title{\textsc{Linked Lists}}
\date{February 22 to February 26, 2016}
%%%%%%%%%%%%%%%%%%%%%%%%%%%%%%%%%%%%%%%%%%%%%%%%%%%%%%%%%%%%%%%%%%%%%%%%%%%%%%%

\begin{document}
\maketitle
\rule{\textwidth}{0.15em}
\fontsize{12}{15}\selectfont

%%% INCLUDE TOPICS HERE %%%%%%%%%%%%%%%%%%%%%%%%%%%%%%%%%%%%%%%%%%%%%%%%%%%%%%%


%%% Question %%%

\begin{blocksection}
For each of the following problems, assume linked lists are defined as follows:
\newline
\begin{lstlisting}
class Link:

    empty = ()

    def __init__(self, first, rest=empty):
        assert rest is Link.empty or isinstance(rest, Link)
        self.first = first
        self.rest = rest
\end{lstlisting}

To check if a \texttt{Link} is empty, compare it against the class attribute \texttt{Link.empty}:
\newline
\begin{lstlisting}
if link is Link.empty:
    print('This linked list is empty!')
\end{lstlisting}
\end{blocksection}

\section{What Would Python Print?}
\begin{questions}
\begin{blocksection}
\question What will Python output? Draw box-and-pointer diagrams to help determine this. 

\begin{lstlisting}
>>> a = Link(1, Link(2, Link(3)))
>>> a.first
\end{lstlisting}
\begin{solution}[.25in]
\begin{lstlisting}
INSERT ANSWER
\end{lstlisting}
\end{solution}

\begin{lstlisting}
>>> a.first = 5
>>> a.first
\end{lstlisting}
\begin{solution}[.25in]
INSERT ANSWER
\end{solution}


\begin{lstlisting}
>>> a.rest.first
\end{lstlisting}
\begin{solution}[.25in]
INSERT ANSWER
\end{solution}

\begin{lstlisting}
>>> a.rest.rest.rest.rest.first
\end{lstlisting}
\begin{solution}[.25in]
\begin{lstlisting}
INSERT ANSWER
\end{lstlisting}
\end{solution}

\begin{lstlisting}
>>> a.rest.rest.rest = a
>>> a.rest.rest.rest.rest.first
\end{lstlisting}
\begin{solution}[.25in]
\begin{lstlisting}
[1, 2, 3]
\end{lstlisting}
\end{solution}

\end{blocksection}

\section{Code Writing Questions}

%%% Question %%%
\begin{blocksection}
\question Write a function \texttt{skip}, which takes in a \texttt{Link} and returns a new \texttt{Link}.

\begin{lstlisting}
def skip(lst):
	"""
	>>> a = link(1, link(2, link(3, link(4, empty))))
	>>> link_to_list(a)
	[1, 2, 3, 4]
	>>> b = skip(a)
	>>> link_to_list(b)
	[1, 3]
	>>> link_to_list(a)
	[1, 2, 3, 4]
	"""
\end{lstlisting}
\begin{solution}[1in]
\begin{lstlisting}
	if l is Link.empty:
		return Link.empty
	if l.rest is Link.empty:
		return l
	return Link(l.first, skip(l.rest.rest))
\end{lstlisting}
\end{solution}
\end{blocksection}

%%% Question %%%

\begin{blocksection}
\question Now write function \texttt{skip} by mutating the original list, instead of returning a new list. Do NOT call the \texttt{Link} constructor.

\begin{lstlisting}
def skip(lst):
	"""
	>>> a = Link(1, Link(2, Link(3, Link(4))))
	>>> b = skip(a)
	>>> b
	Link(1, Link(3))
	>>> a
	Link(1, Link(3)) 
	"""
\end{lstlisting}

\begin{solution}[1in]
\begin{lstlisting}
def skip(lst): # Recursively
	if lst is Link.empty or lst.rest is Link.empty:        
		return lst
	lst.rest = skip(lst.rest.rest)
		return lst

def skip(lst): # Iteratively
	if lst is Link.empty:
        return Link.empty
    original = lst
    while lst.rest is not Link.empty:
        lst.rest = lst.rest.rest
        lst = lst.rest
    return original
\end{lstlisting}
\end{solution}



\end{blocksection}

\end{questions}

%%%%%%%%%%%%%%%%%%%%%%%%%%%%%%%%%%%%%%%%%%%%%%%%%%%%%%%%%%%%%%%%%%%%%%%%%%%%%%%

\end{document}
