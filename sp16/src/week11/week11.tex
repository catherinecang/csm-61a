\documentclass{exam}
\usepackage{../../commonheader}
\lstset{language=Scheme}

%%% CHANGE THESE %%%%%%%%%%%%%%%%%%%%%%%%%%%%%%%%%%%%%%%%%%%%%%%%%%%%%%%%%%%%%%
\discnumber{8}
\title{\textsc{Streams, Iterators, & Generators}}
\date{April 4 to April 9, 2016}
%%%%%%%%%%%%%%%%%%%%%%%%%%%%%%%%%%%%%%%%%%%%%%%%%%%%%%%%%%%%%%%%%%%%%%%%%%%%%%%

\begin{document}
\maketitle
\rule{\textwidth}{0.15em}
\fontsize{12}{15}\selectfont

%%% INCLUDE TOPICS HERE %%%%%%%%%%%%%%%%%%%%%%%%%%%%%%%%%%%%%%%%%%%%%%%%%%%%%%%


%%% Question %%%

\section{Streams}
\begin{questions}
\begin{blocksection}
\question What’s the advantage of using a stream over a linked list?
\begin{solution}[0.2in] 
Lazy evaluation. We only evaluate up to what we need.
\end{solution}

\quesiton What’s the maximum size of a stream?
\begin{solution}[0.2in]
Infinity
\end{solution}

\question What’s stored in first and rest? What are their types? 
\begin{solution}[0.2in]
First is a value, rest is another stream (either a method to calculate it, or an already calculated stream). In the case of Scheme, this is called a promise.
\end{solution}

\question When is the next element actually calculated?
\begin{solution}[.2in]
Only when it's requested (and hasn't already been calculated)
\end{solution}
\end{blocksection}

\section{What Would Scheme Print?}
\begin{blocksection}
\question For each of the following lines of code, write what scheme would output.

\begin{lstlisting}
scm> (define x 1)
\end{lstlisting}
\begin{solution}[.25in]
\texttt{x}
\end{solution}

\begin{lstlisting}
scm> (if 2 3 4)
\end{lstlisting}
\begin{solution}[.25in]
\texttt{3}
\end{solution}

\begin{lstlisting}
scm> (delay (+ x 1))
\end{lstlisting}
\begin{solution}[.25in]
\texttt{#[promise]}
\end{solution}

\begin{lstlisting}
scm> (define (foo x) (+ x 10))
\end{lstlisting}
\begin{solution}[.25in]
\texttt{foo}
\end{solution}

\begin{lstlisting}
scm> (define bar (cons-stream (foo 1) (cons-stream (foo 2) bar)))
\end{lstlisting}
\begin{solution}[.25in]
\texttt{bar}
\end{solution}

\begin{lstlisting}
scm> (car bar)
\end{lstlisting}
\begin{solution}[.25in]
\texttt{11}
\end{solution}
\end{blocksection}

\begin{blocksection}
\begin{lstlisting}
scm> (cdr bar)
\end{lstlisting}
\begin{solution}[.25in]
\texttt{#[promise]}
\end{solution}

\begin{lstlisting}
scm> (define (foo x) (+ x 1))
\end{lstlisting}
\begin{solution}[.25in]
\texttt{foo}
\end{solution}

\begin{lstlisting}
scm> (cdr-stream bar)
\end{lstlisting}
\begin{solution}[.25in]
\texttt{(3 . #[promise])}
\end{solution}

\begin{lstlisting}
scm> (define (foo x) (+ x 5))
\end{lstlisting}
\begin{solution}[.25in]
\texttt{foo}
\end{solution}

\begin{lstlisting}
scm> (car bar)
\end{lstlisting}
\begin{solution}[.25in]
\texttt{11}
\end{solution}

\begin{lstlisting}
scm> (cdr-stream bar)
\end{lstlisting}
\begin{solution}[.25in]
\texttt{(3 . #[promise])}
\end{solution}

\begin{lstlisting}
scm> (define (foo x) (+ x 5))
\end{lstlisting}
\begin{solution}[.25in]
\texttt{foo}
\end{solution}

\begin{lstlisting}
scm> (car bar)
\end{lstlisting}
\begin{solution}[.25in]
\texttt{11}
\end{solution}

\begin{lstlisting}
scm> (cdr-stream bar)
\end{lstlisting}
\begin{solution}[.25in]
\texttt{(3 . #[promise])}
\end{solution}
\end{blocksection}

\section{Code Writing Questions for Streams}

%%% Question %%%
\begin{blocksection}
\question Write out \texttt{double_naturals}, which is a stream that evaluates to the sequence 1, 1, 2, 2, 3, 3, etc.
\begin{lstlisting}
(define (double_naturals)
    (double_naturals_helper 1 0)
)

(define (double_naturals_helper first flag)




)
\end{lstlisting}

\begin{solution}[1in]
\begin{lstlisting}
(define (double_naturals_helper first flag)
    (if (= 1 flag)
        (cons-stream first (double_naturals_helper (+ 1 first) 0))
        (cons-stream first (double_naturals_helper first 1))
    )
)

;Alternative Solutions
(define (double_naturals_helper first flag)
    (cons-stream first (double_naturals_helper (+ flag first) (- 1 flag)))
)
\end{lstlisting}
\end{solution}

\question Write out \texttt{interleave}, which returns a stream that alternates between the values in stream1 and stream2. Assume that the streams are infinitely long.
\begin{lstlisting}
(define (interleave stream1 stream2)





)
\end{lstlisting}

\begin{solution}[1in]
\begin{lstlisting}
(define (interleave stream1 stream2)
(cons-stream (car stream1) 
 (interleave stream2 (cdr-stream stream1)))

(cons-stream (car stream1)
 (cons-stream (car stream2)
  (interleave (cdr-stream stream1)
  (cdr-stream stream2))))
)
\end{lstlisting}
\end{solution}
\end{blocksection}

%%% Question %%%
\section{Iterators}
\begin{blocksection}
\question What is difference between an iterable and an iterator?
\begin{solution}[0.25in]
Iterator: Mutable object that tracks a position in a sequence, advancing on __next__ 

Iterable: Represents a sequence and returns a new iterator on __iter__

To use in an English sentence:
Lists are "iterable". To go through a list, you make an object called an "iterator" to scan through the list.
\end{solution}
\end{blocksection}

\begin{blocksection}
\question \textbf{Accumulator} Write an iterator class that takes in a list and calculates the sum of the list thus far.
\begin{lstlisting}
>>> accu = Accumulator([1, 2, 3, 4, 5, 6])
>>> for a in accu:
            print(a)
1
3
6
10
15
21
\end{lstlisting}
\end{blocksection}

\begin{solution}
\begin{lstlisting}
class Accumulator:
	def __init__(self, lst):
		self.lst = lst
		self.index = 0
		self.sum = 0
	def __next__(self):
		if self.index >= len(self.lst):
			raise StopIteration()
		self.sum += self.lst[self.index]
		self.index += 1
		return self.sum
	def __iter__(self):
		return self
\end{lstlisting}
\end{solution}

\question Is this an iterator or an iterable or both?
\begin{solution}
Both; the __iter__ method returns self and the __next__ method is implemented. Note that an iterator is always an iterable, but an iterable is not always an iterator.
\end{solution}

\question (Optional) Make \texttt{Accumulator} work if it takes in any iterable, not just a list

\begin{solution}
\begin{lstlisting}
class Accumulator:
	def __init__(self, iterable):
    		self.iterable = iterable
		self.iterator = iter(iterable)
            self.sum = 0
	def __next__(self):
    	self.sum += next(self.iterator)
    		return self.sum
	def __iter__(self):
    		return self
\end{lstlisting}
\end{solution}

\section{Generators}

%%% Question %%%
\begin{blocksection}
\question Define \texttt{well-formed}, which determines whether \texttt{lst} is a well-formed list or not. Assume that \texttt{lst} only contains numbers.

\begin{lstlisting}
; Doctests
> (well-formed '())
true
> (well-formed '(1 2 3))
true
; List doesn't end in nil
> (well-formed (cons 1 2))
false
; Nested lists are ok
> (well-formed (cons (cons 1 2) nil))
true
\end{lstlisting}

\begin{solution}[0.75in]
\begin{lstlisting}
; well-formed with a nested if statement
(define (well-formed lst)
    (if (null? lst)
        #t
        (if (number? lst)
            #f
            (well-formed (cdr lst)))))

; well-form with a cond statement
(define (well-formed lst)
    (cond ((null? lst) #t)
        ((number? lst) #f)
        (else (well-formed (cdr lst)))))
\end{lstlisting}
\end{solution}
\end{blocksection}

%%% Question %%%
\begin{blocksection}
\question Define \texttt{is-prefix}, which takes in a list \texttt{p} and a list \texttt{lst} and determines if \texttt{p} is a prefix of \texttt{lst}.

\begin{lstlisting}
; Doctests:
> (is-prefix '() '())
true
> (is-prefix '() '(1 2))
true
> (is-prefix '(1) '(1 2))
true
> (is-prefix '(2) '(1 2))
false
; Note here p is longer than lst
> (is-prefix '(1 2) '(1))
false
\end{lstlisting}

\begin{solution}[0.5in]
\begin{lstlisting}
; Same as below, but with cond
(define (is-prefix p lst)
    (cond ((null? p) #t)
        ((null? lst) #f)
        (else (and (= (car p) (car lst))
            (is-prefix (cdr p) (cdr lst))))))

; Solution that checks if lst is null for the last doctest
(define (is-prefix p lst)
    (if (null? p)
        #t
        (if (null? lst)
            #f
            (and
                (= (car p) (car lst))
                (is-prefix (cdr p) (cdr lst))))))
\end{lstlisting}
\end{solution}
\end{blocksection}

\end{questions}

%%%%%%%%%%%%%%%%%%%%%%%%%%%%%%%%%%%%%%%%%%%%%%%%%%%%%%%%%%%%%%%%%%%%%%%%%%%%%%%

\end{document}
